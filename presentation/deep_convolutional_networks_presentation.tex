%\documentclass[xcolor=dvipsnames,compress,handout]{beamer}
\documentclass[xcolor=dvipsnames,compress]{beamer}
\usepackage[utf8]{inputenc}
\usepackage[ngerman,english]{babel}
\usepackage{pgf}
\usepackage{graphicx}
\graphicspath{ {images/} }
\usepackage[absolute,overlay]{textpos}
\usepackage{xcolor}
\usepackage{comment}
%\usepackage{algorithm2e}
%\usepackage{algorithmic}
\usepackage{xpatch}
\usepackage{tabularx}
\usepackage[export]{adjustbox}
\usepackage{tikz}
\usepackage{pgffor}
\usepackage{blindtext}

% DESIGN
\definecolor{RUBblue}{rgb}{0,0.21,0.38}
\definecolor{AIblue}{rgb}{0,0.57,0.87}

% DO NOT ASK WHAT HAPPENS HERE
% Okay...
\useoutertheme[subsection=false]{miniframes}
\usecolortheme{beaver}
\beamertemplatenavigationsymbolsempty
\setbeamertemplate{footline}{%
    \begin{beamercolorbox}[wd=\paperwidth,ht=0ex,left]{default}
        %\insertauthor\hfill\insertframenumber%
        \colorbox{AIblue}{\vphantom{2cm}\hspace{14cm}}
    \end{beamercolorbox}
}
\setbeamercolor{item}{fg=RUBblue}
\setbeamercolor{title}{fg=Black}
\setbeamercolor{frametitle}{fg=Black}
\setbeamerfont{title}{size=\large}
\setbeamertemplate{itemize items}[square]
\setbeamercolor{section in head/foot}{bg=AIblue}
\setbeamerfont{footline}{size=\fontsize{15}{12}\selectfont}


\xpatchcmd{\itemize}
  {\def\makelabel}
  {\setlength{\itemsep}{0.2cm}\def\makelabel}{}{}

 
\addtobeamertemplate{frametitle}{}{%
\begin{textblock*}{100mm}(1.05\textwidth,0.535cm)
\includegraphics[height=0.975cm,width=0.98cm]{logo-rub.png}
\end{textblock*}}

% COMMANDS
\newcommand\NEWLINECOMMENT[1]{\STATE\STATE/* #1 */}
\newcommand\Only[2]{\only<#1|handout:#1>{#2}}
\newcommand\overlayImage[6]{
	\only<#1|handout:#2>{
		\begin{textblock*}{\textwidth}(#3cm,#4cm)
			\frame{
				\includegraphics[width=#5\textwidth]{#6}
			}
		\end{textblock*}
	}
}


% TITLEPAGE
\title{\textbf{Deep Convolutional Networks}}
\author{Christian Andreas Mielers\\Phil Yannick Schrör}
\institute{Ruhr-University Bochum\\Institute for Neural Computation\\Study Project}
\date{24th of February 2016}
 
\begin{document}
\section{Welcome}
\subsection{Welcome}
\maketitle

\section{Convolutional Neural Networks}
\subsection{Convolutional Neural Networks}

\frame{
	\frametitle{Convolutional Neural Networks}
	\begin{columns}
		\column{0.5\textwidth}
			\begin{itemize}
				\item Learns the weights of convolutional filters
				\item Exploits spatial structure in the input
				\item Convolving entire input with filter implies shared weights
				\item Reduced amount of weights allows lots of filters
				\item Filters specific to color channels
			\end{itemize}
		\column{0.5\textwidth}
			\includegraphics<1>[width=\textwidth]{dense_layer}
			\includegraphics<2>[width=\textwidth]{convolutional_layer}
			\includegraphics<3>[width=\textwidth]{color_mapping}
	\end{columns}
}

\section{GTSRB}
\subsection{GTSRB}

\frame[<+->]{
	\frametitle{Network Structure}
	\begin{textblock*}{1.1\textwidth}(0.5cm,2.9cm)
		\footnotesize{
			\begin{tabularx}{\textwidth}{cp{0.16\linewidth}p{0.4\linewidth}X}
			\textbf{Layer} & \textbf{Type} & \textbf{Configuration} & \textbf{Activation function} \\
			& & & \\
			\hline
			& & & \\
			0 & Convolutional & 100 filters of size $7\times7$ per channel & $\tanh$ \\
			1 & Max Pooling & Pool size $2\times2$ & - \\
			2 & Convolutional & 150 filters of size $4\times4$ per channel & $\tanh$ \\
			3 & Max Pooling & Pool size $2\times2$ & - \\
			4 & Convolutional & 250 filters of size $4\times4$ per channel & $\tanh$ \\
			5 & Max Pooling & Pool size $2\times2$ & - \\
			6 & Dense & 300 neurons & $\tanh$ \\
			7 & Dense & 43 neurons & softmax
			\end{tabularx}
		}
		%TODO Add image of tanh function somewhere
	\end{textblock*}
}

\frame{
	\frametitle{German Traffic Sign Recognition Benchmark}
	\begin{itemize}
		\item German Traffic Sign Recognition Benchmark
		\item What is the task?
		\item Show some images
	\end{itemize}
	%TODO Maybe create three different plots: the first one shows only the results of the simple setup, the second adds the ones with distortions and the third adds the results, which we achieved by using the RELU activation function
}

\frame{
	\frametitle{Simple Setup}
	\begin{itemize}
		\item Describe Simple Setup
		\item Present Results
	\end{itemize}
}

\begin{frame}
	\frametitle{Results on GTSRB}
	\begin{figure}
		\centering
		\includegraphics[width=0.85\textwidth]{gtsrb_results.png}
	\end{figure}
\end{frame}

\frame{
	\frametitle{Input Distortions}
	\begin{itemize}
		\item Mention input distortions
		\item Explain them
		\item Present distortion parameters
		\item Maybe add one or two images before and after the transformations
	\end{itemize}
}

\frame{
	\frametitle{Results with RELU}
	\begin{itemize}
		\item Add RELU image
		\item Present results with RELU activation function
	\end{itemize}
}

\frame[<+->]{
	\frametitle{Missclassified images}
	\begin{figure}
		\centering
		\includegraphics[width=0.85\textwidth]{gtsrb_mistakes/mistake_fussgaenger.png}\\
		\includegraphics[width=0.85\textwidth]{gtsrb_mistakes/mistake_vorfahrtstrasse.png}\\
		\includegraphics[width=0.85\textwidth]{gtsrb_mistakes/mistake_gefahrenstelle.png}
	\end{figure}
	%TODO Maybe choose more significant images
}

\section{Filter Reuse}
\subsection{Filter Reuse}

\frame{
	\frametitle{Filter Reuse}
	\begin{itemize}
		\item How well do the GTSRB filters generalize?
		\item Initialize new network with same structure randomly
		\item Copy GTSRB filters to the new network
		\item Train only the fully connected layers!
	\end{itemize}
}

\frame{
	\frametitle{COIL100}
	\begin{textblock*}{1.1\textwidth}(0.5cm,1.6cm)
		\begin{figure}
			\centering
			\includegraphics[width=\textwidth]{coil100.png}
		\end{figure}
	\end{textblock*}
	\begin{textblock*}{1.1\textwidth}(0.5cm,4.6cm)
		\begin{itemize}
			\item Columbia Object Image Library 100 $\Rightarrow$ COIL100
			\item 100 different objects
			\item Objects turning on a black turntable
			\item One foto each time the object has turned by $5^\circ$
			\item 72 images per object, 7200 images in total
			\item Random separation into 58 training and 14 test images per object
		\end{itemize}
	\end{textblock*}
}

\frame{
	\frametitle{INRIA}
	\begin{itemize}
		\item Describe INRIA dataset
		\item Show image
		\item Show results with reused filters
		\item Show results with original filters
	\end{itemize}
}

\section{Conclusion}
\subsection{Conclusion}

\frame{
	\frametitle{Conclusion}
	\begin{itemize}
		\item Summarize results
	\end{itemize}
}

\section{Questions?}
\subsection{Questions?}

\frame{
	\frametitle{Questions?}
	\begin{figure}[h!]
		\begin{textblock*}{\textwidth}(1.0cm,2.3cm)
			Questions?\\
			\vspace{0.3cm}
			\frame{
				\includegraphics[width=7cm]{fragen.png}
			}
		\end{textblock*}
	\end{figure}
}

\end{document}

