\documentclass[10pt, a4paper]{article}

\usepackage[utf8]{inputenc}
\usepackage{graphicx}
\graphicspath{ {images/} }
\usepackage{mathtools}
\usepackage{amssymb}
\usepackage{amsmath}
\usepackage[ngerman,english]{babel}
\usepackage{cite}
\usepackage{bibgerm}
\usepackage{fullpage}
\usepackage[top=1.5cm,bottom=1.5cm,left=3.5cm,right=2.5cm,headsep=1.5cm,includeheadfoot]{geometry}
\usepackage{tabularx}
\usepackage{caption}
\usepackage{subcaption}
\usepackage{eurosym}
\usepackage{enumitem}
\usepackage{multicol}
\usepackage{tikz}
\usepackage{tkz-euclide}
\usepackage{pgfplots}
\usepackage{pdflscape}
\usepackage{acronym}
\usepackage{blindtext}
\usepackage{ifthen}
\usepackage{setspace}
\usepackage{cancel}
\usepackage{color}
\usepackage{listings}
\usepackage{comment}
\usepackage{xcolor}
\usepackage{colortbl}
\usepackage[parfill]{parskip}

\usepackage{fancyhdr}
\pagestyle{fancy}

\fancyhf{} % clear all
\fancyhead[L]{\leftmark}
\fancyfoot[C]{-- \thepage{} --}
%\setlength{\headheight}{15pt}
\renewcommand{\headrulewidth}{0.5pt}
\renewcommand{\footrulewidth}{0pt}
\setlength{\skip\footins}{0.7cm}

\usetikzlibrary{graphs}
\usetikzlibrary{positioning}

\onehalfspacing
\setlength\parindent{0pt}

%\everymath{\displaystyle}

\allowdisplaybreaks

\definecolor{AI-BLUE}{rgb}{0,0.57,0.87}

% Eigene Befehle
\newcommand\q[1]{\glqq{}#1\grqq{}}
\renewcommand\equiv{\Leftrightarrow}
\newcommand\vertequal[2]{\underset{\underset{#2}{\parallel}}{#1}}
\newcommand\cif{\text{if }}
\newcommand\abs[1]{\left|#1\right|}
\newcommand\norm[1]{\abs{\abs{#1}}}
\newcommand\diff[1]{\text{ d#1}}
\newcommand\av[1]{\left\langle#1\right\rangle}
\newcommand\ev[1]{\mathbb{E}\left(#1\right)}
\newcommand\br[1]{\left(#1\right)}
\newcommand\ubr[2]{\underbrace{#1}_{#2}}
\newcommand\quer[1]{\overline{#1}}
\newcommand\setequal{\overset{!}{=}}
\newcommand\dint{\displaystyle \int}
\newcommand\dsum{\displaystyle \sum}
\newcommand\dprod{\displaystyle \prod}
\newcommand\closedInt[2]{\left[#1,#2\right]}
\newcommand{\checkbox}{\Large \Square \normalsize \hspace{0.4cm}}

\newcommand\myref[1]{\ref{#1} (S. \pageref{#1})}
\newcommand\myrefcomma[1]{\ref{#1}, S. \pageref{#1}}

\newcommand\nsm{Nagel-Schreckenberg-Modell }

\begin{document}

\thispagestyle{empty}

\setlength{\hoffset}{-0.5cm} % center title page

\lstset{
  basicstyle=\small,           % the size of the fonts that are used for the code
  breaklines=true,             % sets automatic line breaking
  captionpos=b,                % sets the caption-position to bottom
  frame=single,                % adds a frame around the code
  keepspaces=true,             % keeps spaces in text, useful for keeping indentation of code (possibly needs columns=flexible)
  numbers=right,               % where to put the line-numbers; possible values are (none, left, right)
  showspaces=false,            % show spaces everywhere adding particular underscores; it overrides 'showstringspaces'
  stepnumber=1,                % the step between two line-numbers. If it's 1, each line will be numbered
  tabsize=4,                   % sets default tabsize to 4 spaces
  xleftmargin=0.14cm		   % sets left margin
}


\begin{titlepage}
    \begin{center}
    \vphantom{0cm}
    \LARGE \textbf{Documentation}\\
    \vspace{3cm}
    \normalsize
    Study Project Documentation \\
    in the Master Study Program \textcolor{AI-BLUE}{[Applied Computerscience]}\\
    at the Ruhr-University Bochum\\
    in the Winter Term 2015/16\\
    \vspace{4cm}
    \huge \textbf{Convolutional Neural Networks} \\
    \vspace{4cm}
    \normalsize
    \textbf{Project Participants}\\
    B. Sc. Christian Andreas Mielers (108 011 204 956)\\
    B. Sc. Phil Yannick Schrör (108 011 214 024)\\
    \vspace{2cm}
    \textbf{Project Supervisor:}\\
    PD Dr. Rolf P. Würtz
    \end{center}
\end{titlepage}

\newpage
\pagenumbering{arabic}
\setcounter{page}{2}

\tableofcontents

\newpage
\section{Introduction}

Paper \cite{2012schmidhuber}

\section{Results}

\addcontentsline{toc}{section}{References}
\bibliography{ref}{}
\bibliographystyle{alpha}

\end{document}